\chapter{見守りシステム}
\label{chap:poordirection}


\section{見守りシステムの概要}
株式会社国建システムと共同で行っている電子見守りシステムは,見守りエリアからの逸脱管理,受信用スマートフォン同士でのビーコン情報の共有という2つの機能がある.

\section{見守りエリアからの逸脱管理}
介護対象者に小型の発信機(iBeacon)を装着し,そこから発せられる電波(ビーコン)をスマートフォンで受信する.
発信機から発せられる電波が介護者に届く範囲を見守りエリアとし,受信ビーコンが一定時間受信できない場合,見守りエリアから外れたとみなしてスマートフォンから警告メッセージを発する.
また,電波を受信した際には見守りエリアに入ったとし,メッセージを発する.介護対象者・発信機が複数の場合も同様に逸脱管理を行う.

\section{ビーコン情報の共有}
受信用スマートフォンが複数ある場合は,スマートフォン同士で発信機の情報を共有し,見守りエリアを拡大させる.
自らの見守りエリアから逸脱した発信機が,他のスマートフォンで検知されていれば見守りは成功していると判断で
きる.
スマートフォン同士で共有する発信機の情報は,

\begin{itemize}
\item タイムスタンプ
\item 各自の見守りエリアにある発信機ID
\item 見守りエリアから逸脱した発信機ID
\end{itemize}

この3つである.これらの情報を共有することで,同じアドホックネットワーク内にある他のスマートフォンが検知している発信機を知ることができ,見守りエリアを拡大することができる.
