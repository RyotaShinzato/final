\chapter{基礎概念}
\label{chap:concept}

\section{アドホックネットワーク}
アドホックネットワークとは,アクセスポイントを必要としない無線で接続できる端末のみで構成されたネットワークである.
直接電波が届き通信可能である場合には,始点ノードから終点ノードへ1ホップでパケット伝達が行われる.電波が直接届かない場合にはノードを経由(マルチホップ)することで終点ノードまでパケットを伝達する.
端末の中継機能を利用してネットワークを構成しているので,アドホックネットワークには基地局インフラが不要である.

\section{AllJoyn}
AllJoyn はQualcommが開発したオープンソースプロジェクトである.デバイス間通信を実現し,製品やアプリケーション間の相互運用を可能にするデバイス通信フレームワー
クを提供している.
Android,iOS,OS X,Linux,Windows7など様々なOSに対応し,Java,C++,C,JavaScriptなどの言語で使用できる.
AllJoyn は近傍デバイスの探索やP2Pネットワークへの接続,セキュリティなどの要素を提供しており,機器やOSに依存せず開発することができる.
