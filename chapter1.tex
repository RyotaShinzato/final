\chapter{はじめに}
\label{chap:introduction}
\pagenumbering{arabic}

\section{背景}
高齢者介護の現場では,徘徊による事故を防止するために,家族や介護者が多大な負担を強いられているのが現状である.
認知症やその疑いがあり,徘徊などで行方不明になったとして警察に届けられている数は9607人に及ぶ.
愛知県大府市で2007年12月,徘徊症状がある認知症の男性が電車にはねられるという事故が発生し,介護者である男性の遺族への賠償判決が出された.

介護者への負担軽減を目的とした電子的・機械的なセンサーシステムはすでにあるが,既存施設への電気設備の敷設が困難,外出先での対応ができないなどの課題がある.
そのため,市販のスマートフォンを受信機とし,低価格の発信機(iBeacon)を利用した電子見守りシステムの構築を株式会社国建システムと共同で行う.

電子見守りシステムでは、発信機と受信用スマートフォンが、1対1,1対多,多対多の組み合わせでの見守りが可能なしくみを開発する.
特に受信用スマートフォンが複数ある場合,スマートフォン同士でアドホックネットワークを構築し,発信機の情報を共有することで見守りエリアの拡大を図る.

\section{研究目的}
そこで本研究では,スマートフォン同士のアドホックネットワークの構築と,情報共有手法の実装を行い,それらの検証を行う.
今回開発する見守りシステムでは介護者それぞれのスマートフォンを用いて見守りを行うことを想定しており,異なる機器やOSでの開発が予想される.そのため,クロスプラットフォームなP2P(peer-to-peer)型デバイス通信フレームワークであるAllJoynを用いる.

\section{本論文の構成}

%\section{Introduction}
