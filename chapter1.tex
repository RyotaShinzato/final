\chapter{はじめに}
\label{chap:introduction}
\pagenumbering{arabic}

\section{背景}
日本では,高齢者人口が増加し少子高齢化社会が加速している.
それに伴い,増加しているのが認知症患者数である.
厚生労働省の調査\cite{厚生労働省}によると2010年時点で200万人程度といわれ,専門家の間では,すでに65歳以上人口の10\%(242万人程度)に達しているという意見もある.

年々増加している認知症患者であるが,介護現場では,徘徊による事故を防止するために,家族や介護者が多大な負担を強いられているのが現状である.
認知症患者の介護者への調査\cite{山梨県}によると,見守りが常に必要な割合が4割にものぼり,徘徊などで行方不明になったとして警察に届けられている数は9607人に及ぶ.
愛知県大府市で2007年12月,徘徊症状がある認知症の男性が電車にはねられるという事故が発生し,介護者である男性の遺族への賠償判決が出された.

これらの現状から,介護者の負担を軽減し,認知症患者の徘徊を防止する仕組みの構築は急務だといえる.
介護者への負担軽減を目的とした電子的・機械的なセンサーシステムはすでにあるが,既存施設への電気設備の敷設が困難,外出先での対応ができないなどの課題がある.
そのため,スマートフォンを受信機とし,低価格の発信機(iBeacon)を利用した電子見守りシステムの構築を株式会社国建システムと共同で行う.

電子見守りシステムでは,発信機と受信用スマートフォンが,1対1,1対多,多対多の組み合わせでの見守りが可能なしくみを開発する.
特に受信用スマートフォンが複数ある場合,スマートフォン同士でアドホックネットワークを構築し,発信機の情報を共有することで見守りエリアの拡大を図る.

\section{研究目的}
本研究では,上述の電子見守りシステムのスマートフォン同士のアドホックネットワークの構築と,情報共有手法の実装を行い,それらの検討を行う.
電子見守りシステムでは介護者それぞれのスマートフォンを用いて見守りを行うことを想定しており,異なる機器やOSでの開発が予想される.
そのため,クロスプラットフォームなP2P(peer-to-peer)型端末通信フレームワークであるAllJoynを用いてアドホックネットワークの構築を行う.

\section{本論文の構成}
本論文の構成について解説する.

本章では,「はじめに」として本研究の背景と目的について述べた.

第二章では,本研究に関する概念や技術について述べる.

第三章では,本研究の背景としてある電子見守りシステムについて述べる.

第四章では,本研究で作成したアプリケーションの設計や仕様,考察にいて述べる.

第五章では,結論と今後の課題について述べる.
%\section{Introduction}
