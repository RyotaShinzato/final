\chapter{結論}
\section{結論}
本研究では,スマートフォン同士のアドホックネットワークの構築と,情報共有手法の実装を目的とした.
検証の結果から,情報共有手法の実装という目的は達成できたといえるが,アドホックネットワークの構築は達成することができなかった.

\section{今後の課題}
検証で得られたデータから,現在のアプリケーションに追加すべき機能や改善点を検討する.

\begin{itemize}
\item アドホックネットワーク \\
スマートフォン同士でのアドホックネットワークを構築し,屋外での見守りにも対応する必要がある.
そのためには,AllJoyn使用環境の検証やAllJoynの拡張をしていく必要があると考えられる.

\item 情報共有手法 \\
今回の設計では,端末が二台しかないという関係上,二台間での情報共有に留まっている.
そのため今後,端末が三台以上での情報共有を想定し設計・実装をし,端末の最大同時接続数の検証などを進めていく必要がある.

\item アプリケーション全般 \\
  周囲にあるiBeaconを全て検知するという設計で開発を行ったが,実際の利用を想定した場合,アプリケーションの画面から検知するiBeaconを指定できるように改良する必要がある.
  
  また,DiscoveryやiBeaconのスキャンの処理をボタンを押下することによって開始しているが,ユーザーが操作することなく自動で処理を開始するよう設計する必要がある.
  
  ビーコン情報の共有に注力してアプリケーションを作成したため,UI部分が洗練されていないため,操作性といった点に関した検証も行いより洗練させていく必要がある.

\item 検証 \\
端末やiBeacon発信機の数を更に増やして,検知できるiBeaconの数などの検証を進めていく必要がある.
また,実際の利用環境での検証や長期間の運用実験を行い,データを収集するべきである.


\end{itemize}



