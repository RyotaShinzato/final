% 参考文献
\def\line{−\hspace*{-.7zw}−}

\begin{thebibliography}{99}
%\bibitem{*}内の * は各自わかりやすい名前などをつけて、
%論文中には \cite{*} のように使用する。
%これをベースに書き換えた方が楽かも。
%書籍、論文、URLによって若干書き方が異なる。
%URLを載せる人は参考にした年月日を最後に記入すること。

\bibitem{texコマンド} LaTexコマンド集 \\ 
\url{http://www.latex-cmd.com/}

\bibitem{厚生労働省}厚生労働省\\
\url{http://www.mhlw.go.jp/kokoro/speciality/detail_recog.html}

\bibitem{山梨県}認知症高齢者の在宅介護における負担の現状\\
\url{http://www.yafo.or.jp/letter/pdf_new/vol186_2.pdf}

\bibitem{アドホック} アドホックネットワーク \\
\url{http://e-words.jp/w/E382A2E38389E3839BE38383E382AFE3838DE38383E38388E383AFE383BCE382AF.html}

\bibitem{allseen}ALLSEEN ALLIANCE \\
\url{https://allseenalliance.org/}

\bibitem{スライド}AllSeen Alliance wiki \\
\url{https://wiki.allseenalliance.org/}

\bibitem{BLE1}BLEとは\\
\url{http://e-words.jp/w/BLE.html}

\bibitem{BLE2}Bluetooth Low Energyとは\\
\url{http://k-tai.impress.co.jp/docs/column/keyword/20110412_438953.html}

\bibitem{iBeacon}iOS:iBeaconについて \\ 
\url{http://support.apple.com/ja-jp/HT202880}

\bibitem{iBeaconを使用してみよう}iOS7でiBeaconを使用してみよう! \\ 
\url{http://www.gaprot.jp/pickup/ios7/vol2/}

\bibitem{aBeacon}aBeacon$\sim$iBeaconをAndroidで受信する$\sim$ \\
\url{http://www.gaprot.jp/pickup/ibeacon/abeacon/}

\bibitem{ARROWS}ARROWS F-05F\\
\url{http://www.fmworld.net/product/phone/f-05f/}

\bibitem{TORQUE}TORQUE G01\\
\url{http://www.kyocera.co.jp/prdct/telecom/consumer/g01/}


\end{thebibliography}
